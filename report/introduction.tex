Within the field of behavioral finance, there are two broad types of sentiment that are analyzed: investor sentiment and textual sentiment.
Investor sentiment is a belief about future cash flows and investment risks that is not justified by the facts at hand~\citep{BakerMalcolm2007ISit}.
As a survey-based approach, it attempts to capture the subjective judgements of individual investors.
Textual sentiment refers to the degree of positivity or negativity in financial texts.
This approach attempts to extract information about investors’ moods from corporate disclosures/filings, media articles, internet messages, and other corpora.
Textual sentiment captures both subjective judgements and objective conditions in financial markets.

While there are some areas of disagreement in the existing literature on the relationship between textual sentiment and financial markets, the existing literature largely agrees that textual sentiment has potentially strong impacts on stock returns, trading volumes, volatility, and abnormal returns~\citep{KearneyColm2014Tsif}.
Negative sentiment exerts downward pressure on market prices, increased trading volume, and weakly predicts market volatility~\citep{TetlockPaul2007GCtI}.
This effect is particularly pronounced during recessions \citep{garciadiego2013SdR} and for smaller companies~\citep{FergusonNickyJ2015MCaS}.
While negative media coverage has a larger effect than positive, during corporate acquisitions, positive media coverage is predictive of acquisition success~\citep{buehlmaier2015role}.

Textual sentiment can also provide clues for trading around market reversals.
Sentiment can predict short-term reversals, temporary increases in volatility, and a shift of capital from riskier stocks into safer bonds~\citep{DaZhi2015TSoA}.
And when sentiment and market returns point in the same direction, the market is likely overreacting and prone to a price reversal~\citep{froot2017media}.
While financial markets are quick to price in inefficiencies, improved measures of textual sentiment hold the promise of providing outsized returns to early movers.
If we can better measure and model textual sentiment, we will have an advantage over other market participants.

Sentiment analysis is a sub-field of natural language processing (NLP).
It aims to measure sentiment in text, audio, and other media. Aspect-Based Sentiment Analysis (ABSA) is a branch of sentiment analysis which deals with both extracting the opinion targets (aspects) and the sentiment expressed towards them.
Given a collection of documents, the first subtask of ABSA is to identify all aspects present in the document.
For example, the news headline “How Kraft-Heinz Merger Came Together in Speedy 10 Weeks” is about the entity ‘Kraft-Heinz.’ The aspect of Kraft-Heinz discussed is 'corporate M\&A activity'.
For the second subtask, we assume the aspect terms are given.
The task is to identify the sentiment expressed towards the aspect.
In the example above, annotators labeled the headline as having slightly positive sentiment (0.214) because the deal was successful and came together quickly.
Thus, ABSA can be formulated as both a classification and regression task.

In this article, we investigate different methods of measuring aspect-based sentiment within the financial domain.
We begin with a pre-trained BERT language model.
Then we fine-tune it on an unlabeled dataset of 559,451 financial headlines from Bloomberg and Reuters.
We evaluate our approach on sub-task 1 of the FiQA 2018 challenge.